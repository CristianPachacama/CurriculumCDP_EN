%%%%%%%%%%%%%%%%%%%%%%%%%%%%%%%%%%%%%%%%%
% Twenty Seconds Resume/CV
% LaTeX Template
% Version 1.1 (8/1/17)
%
% This template has been downloaded from:
% http://www.LaTeXTemplates.com
%
% Original author:
% Carmine Spagnuolo (cspagnuolo@unisa.it) with major modifications by 
% Vel (vel@LaTeXTemplates.com)
%
% License:
% The MIT License (see included LICENSE file)
%
%%%%%%%%%%%%%%%%%%%%%%%%%%%%%%%%%%%%%%%%%

%----------------------------------------------------------------------------------------
%	PACKAGES AND OTHER DOCUMENT CONFIGURATIONS
%----------------------------------------------------------------------------------------

\documentclass[letterpaper]{twentysecondcv} % a4paper for A4

%----------------------------------------------------------------------------------------
%	 PERSONAL INFORMATION
%----------------------------------------------------------------------------------------

% If you don't need one or more of the below, just remove the content leaving the command, e.g. \cvnumberphone{}


\profilepic{david_prof2_cuadr} % Profile picture                  

\cvname{Cristian David \\ Pachacama} % Your name
%\cvjobtitle{K Nair} % Job title/career 
%\cvjobtitle{Mathematical Engineer, \\  Statistics \& OR}
\cvjobtitle{Data Science Consultant}
\cvdate{21 de Septiembre de 1991} % Date of birth
\cvaddress{cristian.pachacama01@gmail.com} % Email address
\cvnumberphone{+593 958873211 } % Phone number
\cvsite{La Floresta, Quito,  Ecuador} % Personal website
%\cvmail{linkedin:cristian-david-pachacama} % Email address
\cvmail{cristianpachacama.netlify.app/}
%----------------------------------------------------------------------------------------

\begin{document}

%----------------------------------------------------------------------------------------
%	 ABOUT ME
%----------------------------------------------------------------------------------------

%\aboutme{Entusiasta por la ciencia de datos. Mi principal área de estudio es la estadística y econometría orientadas análisis de datos, he complementado mis estudios de Matemática con los del uso de diferentes herramientas de software y lenguajes de programación. Actualmente me encuentro desarrollando investigaciones sobre nuevas técnicas de imputación en el área de Geoestadística Espacio-Temporal, así como Análisis de satisfacción con GLMM’s y distribuciones mixtas.} % To have no About Me section, just remove all the text and leave 

\aboutme{Data science enthusiast. My main area of study is statistics and econometrics oriented data analysis, I have complemented my studies in Mathematics with those of the use of different software tools and programming languages. I am currently developing research on new imputation techniques in the area of Spatio-Temporal Geostatistics, as well as Analysis of satisfaction with GLMM.}

%----------------------------------------------------------------------------------------
%	 SKILLS
%----------------------------------------------------------------------------------------

% Skill bar section, each skill must have a value between 0 an 6 (float)
\skills{{Html | CSS | LateX /4.7},{ MemSQL | Hadoop | Spark /3.5},{MySQL | Microsoft SQL S | MongoDB /4.5},{Python | R | Tableau /5.7}}

%------------------------------------------------

% Skill text section, each skill must have a value between 0 an 6
%\skillstext{{Rápido Aprendizaje/5},{Trabajo en Equipo/4},{Proactividad/5}}

%----------------------------------------------------------------------------------------
\Hobbies{
\newline PhD. Pedro Páez
\newline \textbf{Associate Professor (UCE)}
%\newline \textit{Universidad Central del Ecuador}
\newline Telf.: +593 99783 3519
\newline \href{https://www.pedropaez.ec/}{https://www.pedropaez.ec/}
\newline
\newline MsC. Mónica Vargas
\newline \textbf{DB Administrator and Security}
\newline Telf.: +593 98451 6480
}


\makeprofile % Print the sidebar

%----------------------------------------------------------------------------------------
%	 INTERESTS
%----------------------------------------------------------------------------------------

%\section{Intereses}

%Crear soluciones integrales para la empresa tomando como pilar la ciencia de datos.

%----------------------------------------------------------------------------------------
%	 EDUCATION
%----------------------------------------------------------------------------------------

\section{Education}

\begin{twenty} % Environment for a list with descriptions
	\twentyitem{2012-2018}{Mathematical Engineering,{\normalfont Statistics \& Operational Research}}{}
	{\emph{Escuela Politécnica Nacional}}
	%\twentyitem{2004-2010}{Bachiller en Ciencias Generales}{}
	%{\emph{Colegio Fernando Ortíz Crespo}}
	%\twentyitem{<dates>}{<title>}{<location>}{<description>}
\end{twenty}

%% -----------------------------------------
\section{Projects}

\begin{twenty} % Environment for a list with descriptions
	\twentyitem{2017-2018}{Forecast and Impact of extreme low levels of streamflow in hydropower plants.}{[Research Assistant]}
{\emph{Escuela Politécnica Nacional.}}

	\twentyitem{May 2018}{Machine Learning model for forecasting the risk of failure of students of the EPN University.}{[Internship Analyst]}
{\emph{Escuela Politécnica Nacional.}}


\twentyitem{Nov 2014}{Tutorials: Calculus, Vector Calculus and Linear Algebra}{[Tutor]}
{\emph{Escuela Politécnica Nacional | CLAVEMAT }}

	%\twentyitem{<dates>}{<title>}{<location>}{<description>}
\end{twenty}

%----------------------------------------------------------------------------------------
%	 PUBLICATIONS
%----------------------------------------------------------------------------------------

\section{Courses \& Certifications}

\begin{twentyshort} % Environment for a short list with no descriptions
    
    %\twentyitem{2018}{Proficiency in the English  language.}{}{\emph{CEC-EPN}}

    \twentyitem{Sep 2019}{Preparation of action plans based on the PMBOK\textregistered \ guide}{}{\emph{Pontificia Universidad Católica del Ecuador}}
    
	\twentyitem{Jul 2019}{Introduction to TensorFlow for Artificial Intelligence, ML and DL}{}
	{\emph{COURSERA}}
	
	\twentyitem{Oct 2018}{Big Data with Apache Spark (by Michael Page)}{}
	{X Conference on Computer Systems and Computing Systems (JISIC), Big Data \& Information Systems. \\ 
	\emph{Escuela Politécnica Nacional}
	}
	
	\twentyitem{Dic 2017}{Big Data in R}{}
	{IV Meeting of Economics. \\ 
	\emph{Escuela Politécnica Nacional}
	}
	
	\twentyitem{Mar 2017}{Introduction to Data Analytics}{}
	{
	\emph{Empresa Pública YACHAY EP}
	}
	
	\twentyitem{Dic 2016}{Using Python to Access Web Data}{}
	{University of Michigan. \\ 
	\emph{COURSERA}
	}
	
	\twentyitem{Oct 2016}{Sample Design Course for Business Surveys}{}
	{XV Meeting of Mathematics and its Applications. \\ 
	\emph{Escuela Politécnica Nacional}
	}
	
	%\twentyitem{Sep 2016}{Introducción a Data Science: Programación Estadística con R}{}
	%{UNAM \\ 
	%\emph{COURSERA}
	%}
	
	\twentyitem{Sep 2016}{II Latin American School of Algorithms}{}
	{MODEMAT (EPN) y Núcleo Milenio 
	%\\ 
	%\emph{Escuela Politécnica Nacional}
	}
	
	\twentyitem{Jul 2014}{R Programming}{}
	{XIV Meeting of Mathematics and its Applications (EPN). 
	%\\ 
	%\emph{Escuela Politécnica Nacional}
	}

	
	% \twentyitem{Jul 2014}{Curso Tutorial de Programación en R}{}
	% {XIV Encuentro de Matemática y sus Aplicaciones. \\ 
	% \emph{Escuela Politécnica Nacional}
	% }

	%\twentyitemshort{1865}{Chapter Four,  The Rabbit Sends a Little Bill.}
	%\twentyitemshort{1865}{Chapter Five,  Advice from a Caterpillar.}
	%\twentyitemshort{<dates>}{<title/description>}
	
\end{twentyshort}

%----------------------------------------------------------------------------------------
%	 AWARDS
%----------------------------------------------------------------------------------------

\section{Recognitions}

\begin{twentyshort} % Environment for a short list with no descriptions
	\twentyitemshort{Jul 2018}{Ambassador for Research and Science.\\\emph{Research Leap} }
	%\twentyitemshort{<dates>}{<title/description>}
\end{twentyshort}

%----------------------------------------------------------------------------------------
%	 EXPERIENCE
%----------------------------------------------------------------------------------------

\section{Experience}

\begin{twenty} % Environment for a list with descriptions
	
	\twentyitem{2019-2020}{Data Scientist}{}
	{MOBILVENDOR - MIVSHELL}
	
	\twentyitem{2019-2020}{Business Intelligence  Analyst}{}
	{RECOVER Ecuador}
	
	\twentyitem{2018-2019}{Project Manager \\
	Institute of Economic Research}{}
	{Pontitificia Universidad Católica del Ecuador}
	
	\twentyitem{Abr 2018}{INSTRUCTOR: Dynamic and Reproducible Reporting with Rmarkdown.\\
	Free Software School, tools for all Data Scientists.
	}{}
	{Escuela Politécnica Nacional | FEPON | MATH Consultants}
	
	
	
\end{twenty}

%----------------------------------------------------------------------------------------
%	 OTHER INFORMATION
%----------------------------------------------------------------------------------------

%\section{Other information}

%\subsection{Review}

%Alice approaches Wonderland as an anthropologist, but maintains a strong sense of noblesse oblige that comes with her class status. She has confidence in her social position, 

%----------------------------------------------------------------------------------------
%	 SECOND PAGE EXAMPLE
%----------------------------------------------------------------------------------------

%\newpage % Start a new page

%\makeprofile % Print the sidebar

%\section{Other information}

%\subsection{Review}

%Alice approaches Wonderland as an anthropologist, but maintains a strong sense of noblesse oblige that comes with her class status. She has confidence in her social position, education, and the Victorian virtue of good manners. Alice has a feeling of entitlement, particularly when comparing herself to Mabel, whom she declares has a ``poky little house," and no toys. Additionally, she flaunts her limited information base with anyone who will listen and becomes increasingly obsessed with the importance of good manners as she deals with the rude creatures of Wonderland. Alice maintains a superior attitude and behaves with solicitous indulgence toward those she believes are less privileged.

%\section{Other information}

%\subsection{Review}

%Alice approaches Wonderland as an anthropologist, but maintains a strong sense of noblesse oblige that comes with her class status. She has confidence in her social position, education, and the Victorian virtue of good manners. Alice has a feeling of entitlement, particularly when comparing herself to Mabel, whom she declares has a ``poky little house," and no toys. Additionally, she flaunts her limited information base with anyone who will listen and becomes increasingly obsessed with the importance of good manners as she deals with the rude creatures of Wonderland. Alice maintains a superior attitude and behaves with solicitous indulgence toward those she believes are less privileged.

%----------------------------------------------------------------------------------------

\end{document} 
